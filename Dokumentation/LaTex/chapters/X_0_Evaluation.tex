\section{Evaluation} \label{sec:Evaluation}

\subsection{Testkonzept}
\todoin{Bogi, Vic - lest euch das bitte durch und guckt ob wir das wirklich so verteidigen können.. thx}

	Ziel der durchgeführten Tests war es, mögliche Fehlerzustände im implementierten Programm aufzudecken und außerdem zu belegen, dass das System
	wie erwartet funktioniert. 
	
	% White Box
	% \todoin{Entwicklertest, Testbarkeit der Software (Treiber,...), Positiv- und Negativtests,}
	Da die Verwendung strukturorientierter Testverfahren die Entwicklung eines Testrahmens bedeuten und dies den Rahmen des Projektes sprengen würde,
	wurde von der Verwendung dieser Verfahren abgesehen.  
	Stattdessen wurde der Einsatz verschiedener Black-Box Methoden vorgesehen.
	
	% Black Box
	% \subsection{Funktionaler Test}
	Um zu überprüfen ob das entwickelte Spiel die gewünschten Funktionen bietet wurden funktionale Tests durchgeführt. 
	Als Spezifikation hierfür wurden die in Abschnitt \ref{sec:Spielregeln} beschriebenen Eigenschaften des Spiels \textit{Schiffe-Versenken}
	verwendet. 
	
	\todoin{zustandsbasierter test - testet auch ungültige zustandsübergänge - möglcih???}
	
	% nicht funktional
	Außerdem wurden auch nicht funktionale Testverfahren verwendet. Zum Einen wurde ein Langzeittest durchgeführt, bei dem zwei
	Prolog-Clients 1000 Spiele gegen einander Spielen. Ziel dieses Tests war es diese hohe Anzahl von Spielen fehlerfrei zu beenden. 	
	
	\todoin{nicht funktional: portabilitätstest - wurde ja auf 2 plattformen entwickelt??!}
	\todoin{nicht funktional: Benutzbarkeit - hat nicht jemand von euch lust seine mama mal spielen zu lassen? oder ich setz beate ran wenn die mal hier ist ;)}

\subsection{Testfälle}

	Vorbedignung - KI Ausgabe auf der Konsole
	
	- eigene Schiffe gültig platziert
	- richtige anzahl und größe der schiffe
	- gegnerische schiffe werden attackiert wenn ein treffer gelandet wurde
	- gegnerisches schiff wird nach versenken mit wasser umschlossen
	- 