\section{Spielregeln}
\label{sec:Spielregeln}

Im folgenden werde die Spielregeln des Spiels \emph{Schiffe-Versenken} beschrieben. Diese allgemein gültigen Regeln, sind für 
die im folgenden beschriebene Umsetzung des Spiels in Prolog und Java als Spezifikation anzusehen.

%Da die allgemeinen Regeln des Spiels \textit{Schiffe-Versenken} bekannt sein sollten, werden an dieser Stelle nur die relevanten Eckdaten 
%und besondere Vereinbarungen aufgeführt:

\todoin{Findet jemand Fehler, Mehrdeutigkeiten, Ungenauigkeiten, Auslassung?!- SB}

\begin{itemize}
	\item \textbf{Spielfeld} \newline Für das eigene und das gegnerische Spielfeld gelten die folgenden Regeln:
		\begin{itemize}
			\item Das Spielfeld eines jeden Spielers ist 10x10 Felder groß.
			\item Jedes eigene Feld hat einen der folgenden Status: Wasser, Schiff, Getroffen oder Versenkt
			\item Jedes gegnerische Feld hat infolgedessen einen der folgenden Status: Unbekannt, Wasser, Treffer oder Versenkt.
		\end{itemize}
	\item \textbf{Schiffe} \newline Für die am Spiel beteiligten Schiffe gelten die folgenden Regeln:
		\begin{itemize}
			\item Folgende Schiffe müssen auf dem Spielfeld platziert werden: 1x 5er, 1x 4er, 2x 3er, 1x 2er.
			\item Schiffe dürfen einander nicht berühren.
			\item Schiffe dürfen Horizontal und Vertikal, nicht aber Diagonal auf dem Spielfeld platziert werden. % Fragt sich, ob Sie AKTROPHAL positioniert werden können ;-) - kleiner Scherz ... höhö
			\item Schiffe dürfen Diagonal versetzt (Aneinander liegende Eckpunkte) platziert werden.
			\item Schiffe dürfen nach der ersten Platzierung nicht mehr verschoben werden.
		\end{itemize}
	\item \textbf{Spielablauf} \newline Für den Spielablauf gelten die Folgenden Regeln und abläufe:
		\begin{itemize}
			\item Beide Spieler 'schießen' abwechselnd auf das Spielfeld des anderen.
			\item Das Los entscheidet, welcher Spieler beginnt.
			\item Ein angreifender Spieler teilt dem Gegner die angegriffene Koordinate mit (bspw. "E5").
			\item Der angegriffene Spieler Antwortet nun wahrheitsgemäß mit:
				\begin{itemize}
					\item \emph{"Wasser!"} wenn kein Schiff auf dem angegriffenen Feld steht.
					\item \emph{"Treffer!"} wenn ein Schiff auf dem angegriffenen Feld steht.
					\item \emph{"Versenkt!"} wenn der letzte Teil eines Schiffes auf diesem Feld getroffen wurde.
				\end{itemize}
			\item \textbf{Spielende:} Es gewinnt der Spieler, der zuerst alle Fünf Schiffe des Gegners versenkt hat.
		\end{itemize}
\end{itemize}