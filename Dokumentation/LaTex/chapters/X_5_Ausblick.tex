\section{Fazit und Ausblick} \label{sec:Ausblick}
	\todoin{was fehlt noch? bekannte Fehler?}
	\todoin{Fazit der Teilnehmer??? vllt bzgl. Prolog?}
	\todoin {Ausblick: Ja! Fazit d Teilnehmer: Nein! -SB}

	Das entwickelte Programm ermöglicht das Spielen von \textit{Schiffe-Versenken} unter anderem gegen \textit{den Computer}. 
	Landet die entwickelte Intelligenz einen Treffer, so ist sie in der Lage das zugehörige Schiff vollständig zu versenken. 
	Außerdem werden weitere Schüsse auf die unmittelbare Umgebung eines versenkten Schiffes verhindert, da sich hier auf keinen Fall ein
	weiteres Schiff befinden kann.
	
	Als Erweiterung des Programms könnte man einige Felder für den Beschuss ausschließen, wenn die kleinen Schiffe bereits versenkt wurden.
	Geht man beispielsweise davon aus, dass das kleinste Schiff (\textit{der 2er}) vollständig versenkt wurde, so kann man alle 
	Lücken, die lediglich Platz für dieses kleinste Schiff bieten, von den weiteren Überlegungen ausnehmen. Somit reduziert sich die 
	Anzahl der möglichen Felder für einen Beschuss und die Wahrscheinlichkeit für einen Treffer erhöht sich.

	Als weitere Verbesserung ist es denkbar, die zufälligen Angriffe (bei leerer Openlist) durch ein strukturiertes Angreifen zu ersetzen.
	Ein mögliches Vorgehen hierbei wäre es, jedes vierte gegnerische Feld anzugreifen und dieses Muster in jeder Zeile um ein Feld zu verschieben.  
	Auf diese Weise ist das Treffen der beiden größten Schiffe garantiert. Trifft dieses Vorgehen durch Zufall nicht auch die anderen 
	Schiffe, so wird beim weiteren Vorgehen jedes zweite gegenerische Feld attackiert. Auf diese Weise werden auch die übrigen Schiffe getroffen.
	\todoin {versteht ihr dass so?! -SB \newline (Sobald man ein Schiff am Rand trifft, schließt sich Automatisch eine Richtung der Schiffsorientierung aus!)}
	Bei der Platzierung von Schiffen kann eine Strategie angewendet werden, welche es vermeidet Schiffe am Rand des Spielfeldes zu platzieren. 
	Hierdurch würden sich die Schussmöglichkeiten für den Gegner nicht noch zusätzlich durch den Spielfeldrand einschränken.
	
		
	\todoin{lernende software, analyse des gegners}


