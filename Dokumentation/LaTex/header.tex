%!TEX root = /Users/stefanbogdanski/Dropbox/BÄTTLESHÖP/Dokumentation/dokumentation.text
% 
% Apostel, Bogdanski, Ritter
%
% Wahlpflichtfach Künstliche Intelligenz:
% Projekt - Schiffe Versenken
%
% Hochschule Bremen - University of applied siences
% ============================================================================
%
% header.tex
%
% Präambel und LaTeX Einstellungen

\documentclass[
    pdflatex,           % Use PDFTex
    a4paper,            % A4 paper
    oneside,            % oneside print
    12pt,               % Font size 12pt
    captions=tableheading,  % korrekte Abstaende bei TabellenUEBERschriften
    chapterprefix,      % Chapter write to as chapter
    headsepline,        % Line after header
    footsepline,        % Line before footer
    headinclude,        % Kopfzeile wird Seiten-Layouts mit beruecksichtigt
    footinclude=false,  % Fu{\ss}zeile wird Seiten-Layouts mit 
    plainheadsepline,   % horizontale Linie auch beim plain-Style
    fleqn,              % Write the formula left-aligned
    appendixprefix,
	bibtotocnumbered,	%referencen mit nummerieren und ins inhalts VZ 
	dvipsnames
]{scrartcl}
\usepackage{graphicx}
\usepackage{graphics}
\usepackage{epsf}
\usepackage{epsfig}
\usepackage{amssymb}
\usepackage{amsmath}
%
% package for the appendix.
%
\usepackage{appendix}

% mehrzeilige Captions ausrichten
\usepackage[font=footnotesize]{caption}
%
% package to create subfigure, subtables
%
\usepackage{subfigure} 

%
%package to include pdf-files to Latex Document
%
\usepackage{pdfpages}

%
% package for the index.
%
\usepackage{makeidx} 

\usepackage{rotfloat}



%
% using babel
%
\usepackage[ngerman]{babel}

%
% package to use the UTF8 character and their writings
%
\usepackage[utf8]{inputenc}
%\usepackage[T1]{fontenc}
\usepackage{textcomp}

%
% Flie{\ss}umgebung: Erm\"{o}glicht Platzierung [H] f\"{u}r Grafiken und Tabellen
%
\usepackage{float}
\restylefloat{figure}
\restylefloat{table}

%
% package for more table settings
%
\usepackage{array}%,tabularx}
%\usepackage[table,svgnames]{xcolor}
\usepackage{ragged2e}

\usepackage{setspace}            % Zeilenabstand einstellbar
\onehalfspacing                  % eineinhalbzeilig einstellen

\typearea[current]{current}        % Neuberechnung des Satzspiegels mit alten Werten nach \"{A}nderung von Zeilenabstand,etc


%
% header und footer
%
%\pagestyle{headings}
%\usepackage{scrpage2}            % Kopf und Fusszeilen-Layout
\renewcommand{\headfont}{\normalfont\sffamily}    % Kolumnentitel serifenlos
\renewcommand{\pnumfont}{\normalfont\sffamily}    % Seitennummern serifenlos
%\pagestyle{headings}
%\ihead[]{\headmark}              % Kolumnentitel immer oben innen
%\ohead[\pagemark]{\pagemark}     % Seitennummern immer oben aussen
%\ofoot{\pagemark}                       % Seitennummern in der Fusszeile loeschen

\usepackage[automark]{scrpage2}
\ihead{}
\ohead{\headmark}
\chead{}
\cfoot{}
\ofoot{\pagemark}

\pagestyle{scrheadings}

% 
\usepackage{blindtext}
% define the color "LinkColor"
%
\definecolor{LinkColor}{rgb}{1,0,0}
\definecolor{CiteColor}{rgb}{0,1,0}
\definecolor{URLColor}{rgb}{0,0,1}

%
% package for using Hyperlinks in PDF files
%
 \usepackage[
	pdftitle={Projekt - Schiffe Versenken},
	pdfauthor={Apostel, Bogdanski, Ritter},
	pdfsubject={Wahlpflichtfach KI},
	plainpages=true
    ]{hyperref}
\hypersetup{colorlinks=false,
	linkcolor=LinkColor,
	citecolor=CiteColor,
	filecolor=LinkColor,
	menucolor=LinkColor,
%	pagecolor=LinkColor,
	urlcolor=URLColor}

%
% package to represent listings in their formats
%

\usepackage[colorinlistoftodos]{todonotes}
\newcommand{\todoin}[1]{\todo[inline]{#1}}

%
% package to use the abstract environment
%
\usepackage{abstract}

%
% package to use URLs
%
\usepackage{url}
\usepackage{listings} 
%\usepackage{color} 
%\usepackage[dvipsnames]{xcolor}  

%
% build index
%
\makeindex

%
% build glossary
%
\makeglossary

%\usepackage[savemem]{listings}


\usepackage{longtable}