%!TEX root = /Users/stefanbogdanski/Dropbox/BÄTTLESHÖP/Dokumentation/dokumentation.tex
%
% Apostel, Bogdanski, Ritter
%
% Wahlpflichtfach Künstliche Intelligenz:
% Projekt - Schiffe Versenken
%
% Hochschule Bremen - University of applied siences
% ============================================================================
%
% 0_deckblatt.tex
%
% Beschreibung des Dateiinhalts
\begin{titlepage}
	\begin{figure}[lt] % (fold)
		\includegraphics[scale=0.5]{images/logo_a4.pdf}
	\end{figure}
	% figure figure label (end)
	\begin{Center}
		\Huge
		Wahlpflichtfach KI\newline
		\vspace*{1cm}
		\Large
		Projektdokumentation: Schiffe Versenken
	\end{Center}	
	\vfill
	\begin{table}[H] % (fold)
		\centering
		\begin{tabular}{ll}
			Autoren: & Victor Apostel, Stefan Bogdanski, Sibille Ritter \\
			Studiengang: & (Internationaler) Studiengang Technische Informatik B.Sc. \\
		\end{tabular}
		\label{label}
	\end{table}
	\todoin {Liebe Gruppenmitglieder, \newline so langsam müssen wir auch bedenken, dass dieses Dokument nicht den Rahmen sprengen darf. m.E. ist es bereits viel zu lang! Der Inhalt ist - wie zu erwarten - auf einem sehr schönen qualitativen Niveau. Aber man darf es hier nicht aus den Fugen laufen lassen. Denkt daran, dass der 'arme' Prof. Breymann das auch alles lesen muss. Wir haben inzwischen ein fast 40(!!!) seitiges Pamphlet zu einer Implementierung von Schiffe versenken verfasst.\newline Ich bin mir bewusst, dass mein Anteil an diesem Dokument der geringste ist, und ich will mich bei Leibe nicht darüber beschweren, dass soviel von euch geschrieben wurde! Aber trotzdem möchte ich apellieren, dass der Rahmen m.E. bereits überschritten ist und nun die Zeit gekommen ist auf die bremse zu treten. Dies ist keine Bachelor-Thesis, dies ist ein WPM. LG, HEGDL und alles! - SB}
\end{titlepage}